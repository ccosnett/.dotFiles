
%%%%%%%%%%%%%%%%%%%%%%%%%%%%%%%%%%%%%%%%% 
% Short Sectioned Assignment
% LaTeX Template
% Version 1.0 (5/5/12)
%
% This template has been downloaded from:
% http://www.LaTeXTemplates.com
%
% Original author:
% Frits Wenneker (http://www.howtotex.com)
%
% License:
% CC BY-NC-SA 3.0 (http://creativecommons.org/licenses/by-nc-sa/3.0/)
%
%%%%%%%%%%%%%%%%%%%%%%%%%%%%%%%%%%%%%%%%%

%----------------------------------------------------------------------------------------
%	PACKAGES AND OTHER DOCUMENT CONFIGURATIONS
%----------------------------------------------------------------------------------------

\documentclass[paper=a4, fontsize=11pt]{scrartcl} % A4 paper and 11pt font size

\usepackage[T1]{fontenc} % Use 8-bit encoding that has 256 glyphs
\usepackage{fourier} % Use the Adobe Utopia font for the document - comment this line to return to the LaTeX default
\usepackage[english]{babel} % English language/hyphenation
\usepackage{amsmath,amsfonts,amsthm} % Math packages

\usepackage{lipsum} % Used for inserting dummy 'Lorem ipsum' text into the template

\usepackage{sectsty} % Allows customizing section commands
\allsectionsfont{\centering \normalfont\scshape} % Make all sections centered, the default font and small caps

\usepackage{fancyhdr} % Custom headers and footers
\pagestyle{fancyplain} % Makes all pages in the document conform to the custom headers and footers
\fancyhead{} % No page header - if you want one, create it in the same way as the footers below
\fancyfoot[L]{} % Empty left footer
\fancyfoot[C]{} % Empty center footer
\fancyfoot[R]{\thepage} % Page numbering for right footer
\renewcommand{\headrulewidth}{0pt} % Remove header underlines
\renewcommand{\footrulewidth}{0pt} % Remove footer underlines
\setlength{\headheight}{13.6pt} % Customize the height of the header

\numberwithin{equation}{section} % Number equations within sections (i.e. 1.1, 1.2, 2.1, 2.2 instead of 1, 2, 3, 4)
\numberwithin{figure}{section} % Number figures within sections (i.e. 1.1, 1.2, 2.1, 2.2 instead of 1, 2, 3, 4)
\numberwithin{table}{section} % Number tables within sections (i.e. 1.1, 1.2, 2.1, 2.2 instead of 1, 2, 3, 4)

\setlength\parindent{0pt} % Removes all indentation from paragraphs - comment this line for an assignment with lots of text

%----------------------------------------------------------------------------------------
%	TITLE SECTION
%----------------------------------------------------------------------------------------

\newcommand{\horrule}[1]{\rule{\linewidth}{#1}} % Create horizontal rule command with 1 argument of height



\begin{document}



\section{Early Education}
\textit{Please provide any additional information relevant to your academic background which should include the name, location and dates of any training courses attended.}

\subsection{School Years}

\begin{enumerate}
  \item I was interested in simple electronics as a school boy. I made robots that navigated my room. Later on I made an automatic physiotherapy machine for the had and printed self designed circuit boards using chemicals and a laser printer in my room at home. I used a 3D printer at school to print the gear box for the hand physio machine. I designed and soldered the electronic components from first principles.

  \item I was keen on playing the piano as a teenager and did music for my leaving cert. I feel that the hand-eye co-ordination necessary for playing the piano transfers quite well to computer programming. I achieved piano grade 6 in 2009.
    
  \item I enjoyed maths at school and I went for extra voluntary maths tutorials on Saturdays at Maynooth college. I took part in Irish maths Olympiads in Limerick and came 14th.
    
  \item I designed and made a custom wooden desk for doing my electronics at home using the Solid Works CAD system. This desk was also equipped with a homemade air extraction system.
  
\end{enumerate}





\subsection{University Years}
 During spare time taught myself to utilize the following software packages:
\begin{itemize}
\item \textbf{Solid Work}  (Computer Aided Design) 2011
\item \textbf{Sibelius 6} (Music Notation and Composition) 2012
\end{itemize}

Spent the following approximate number of hours learning the following computer languages between the years 2012 and 2016:
\begin{itemize}
\item \textbf{Python:}             100 hours
\item \textbf{MATLAB:}            1000 hours
\item \textbf{Mathematica:}       3500 hours
\item \textbf{C:}                   50 hours
\item \textbf{Haskell:}            100 hours
\item \textbf{Prolog:}             300 hours
\item \textbf{LaTeX:}              200 hours
                 
\item \textbf{EMACS with AUCTEX:}  200 hours  

\end{itemize}

During my university year in 2015 and especially in 2016 (when I dropped out of University for a year following a computer game addiction problem and various temporary psychological hassles) I reformed my lifestyle and began learning various computer languages. I completed several MOOCs oneline:
\subsection{MOOCs}

\begin{itemize}

\item \textbf{Machine Learning:} Andrew Ng (Coursera)
  \item  \textbf{Introduction to Programming with MATLAB:} Vanderbilt University
  \item   \textbf{Introduction to Computer Science with Python:} MIT (EDX)

    

    
\end{itemize}

Other MOOCs I Watched the lectures but not actually complete:


\begin{itemize}
\item \underline{Introduction to Mathematical Thinking}: Keith Devlin (Coursera)
\item \underline{Machine Learning}: Pedro Domingos (Coursera)
\item \underline{Introduction to Logic}: Michael Genesereth (Coursera)
\item \underline{Artificial Intelligence}: (Berkeley) Dan Klein, Peter Abbeel (EDX)
\item \underline{Differential Equations}: Paul Blanchard (Coursera)
\end{itemize}

I have always read about the history and developement of science, Computing and Astronomy since my boyhood. I thus became familiar with the life stories and work of the following scientists:

\begin{itemize}
\item{\textbf{Mathematicians:}}
  \subitem{Srinivasa Ramanujan}
  \subitem{Ludwig Wittgenstein}
  \subitem{Paul Erdos}
  \subitem{Terence Tao}
  \subitem{John Nash}
  \subitem{John Conway}
  \subitem{Gregori Perelman}
\item{\textbf{Inventors and Engineers:}}
  \subitem{Elon Musk}
  \subitem{Christof Koch}
    
\item{\textbf{Neuroscientists:}}
  \subitem{Sebastian Seung}
  \subitem{Santiago Ramón y Cajal}
  \subitem{Christof Koch}
  \subitem{Warren McCulloch} 
  \subitem{Walter}
\item{\textbf{Astronomers:}}
  \subitem{Carl Sagan}
  \subitem{Johannes Kepler}
  \subitem{Galileo Galilei}
  \subitem{Jocelyn Bell}
  \subitem{Christian Huygens}
\item{\textbf{Computer Pioneers:}}
  \subitem{Alan Turing}
  \subitem{Charles Babbage}
  \subitem{Ada Lovelace}
  \subitem{Douglas Englebart}
  \subitem{JCR Licklider}
  \subitem{Ken Thompson}
  \subitem{Dennis Ritchie}
\item{\textbf{Physicists:}}
  \subitem{Niels Bohr}
  \subitem{Albert Einstein}
  \subitem{Gottfried Liebnitz}
  \subitem{Richard Feynman}
  \subitem{John Bell}
  \subitem{James Clerk Maxwell}
  \subitem{Max Planck}
  \subitem{Erwin Schrodinger}
  \subitem{Albert Michelson}
\item{\textbf{Artificial Intelligence Pioneers:}}
  \subitem{Yoshua Bengio}
  \subitem{Andrej Karpathy}
  \subitem{Geoffrey Hinton}
  \subitem{Demis Hassabis}
  \subitem{Pedro Domingos}
  \subitem{Judea Pearl}
  \subitem{Ilya Sutskever}
  \subitem{Marvin Minsky}
  \subitem{John McCarthy}
\end{itemize}

I try and contribute to the \underline{Mathematica Community on Stack Overflow}: \textbf{2442 points}\\

I am challenged by solving programming problems on \underline{Project Euler}: I have solved \textbf{87}



\end{document}